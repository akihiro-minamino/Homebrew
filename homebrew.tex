\documentclass[11pt, oneside]{article}   	% use "amsart" instead of "article" for AMSLaTeX format
\usepackage{geometry}                		% See geometry.pdf to learn the layout options. There are lots.
\geometry{letterpaper}                   		% ... or a4paper or a5paper or ... 
%\geometry{landscape}                		% Activate for rotated page geometry
%\usepackage[parfill]{parskip}    		% Activate to begin paragraphs with an empty line rather than an indent
\usepackage{graphicx}				% Use pdf, png, jpg, or eps§ with pdflatex; use eps in DVI mode
								% TeX will automatically convert eps --> pdf in pdflatex		
\usepackage{amssymb}

%SetFonts

%SetFonts


\title{Homebrew}
\author{Akihiro Minamino}
%\date{}							% Activate to display a given date or no date

\begin{document}
\maketitle

\section{Homebrewとは?}
HomebrewはMacOS環境におけるディフェクトスタンダートなパッケージマニュージャーである。\\

\section{Homebrewのページ}
\verb|https://brew.sh/index_ja|\\

\section{よく使うコマンド}
\begin{itemize}
\item brew update: HomebrewのformulaeとHomebrew自体をアップデート\footnote{formulaeとはHomebrewにおけるビルド方法・手順が書かれスクリプト}
\item brew upgrade: 更新があるパッケージを再ビルド
\item brew doctor: 諸問題を検出
\item brew list: インストール済みのパッケージの一覧を見る
\item brew install [formula name]: [formula name]をインストール
\item brew uninstall [formula name]: [formula name]をアンインストール
\item brew info [formula name]: [formula name]の情報を見る
\item brew --config: Homebrewに関わるローカル設定を確認する
\end{itemize}


%\subsection{}



\end{document}  